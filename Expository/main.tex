\documentclass[11pt, reqno]{article}

\usepackage{amsfonts,latexsym,amsthm,amssymb,amsmath,amscd,euscript,bm}
\usepackage[sc]{mathpazo}
\usepackage[margin = 2cm]{geometry}
\usepackage{enumitem}
\usepackage{hyperref}
% sets numbering of enumerate to a, b, c, ...
\renewcommand{\theenumi}{\alph{enumi}}

% Theorems, propositions, etc.
\newtheorem{theorem}{Theorem}
\newtheorem{proposition}[theorem]{Proposition}
\newtheorem{lemma}[theorem]{Lemma}
\newtheorem{corollary}[theorem]{Corollary}

\theoremstyle{definition}
\newtheorem{definition}[theorem]{Definition}
\newtheorem*{claim}{Claim}

\theoremstyle{remark}
\newtheorem*{remark}{Remark}
\newtheorem*{notation}{Notation}

% Allows us of widecheck since loading mathabx causes problems 
\DeclareFontFamily{U}{mathx}{\hyphenchar\font45}
\DeclareFontShape{U}{mathx}{m}{n}{
      <5> <6> <7> <8> <9> <10>
      <10.95> <12> <14.4> <17.28> <20.74> <24.88>
      mathx10
      }{}
\DeclareSymbolFont{mathx}{U}{mathx}{m}{n}
\DeclareFontSubstitution{U}{mathx}{m}{n}
\DeclareMathAccent{\widecheck}{0}{mathx}{"71}
\DeclareMathAccent{\wideparen}{0}{mathx}{"75}

% Examples put into boxes, just an aesthetic choice
% {examplebox} for a single example, {example} is a list environment (\item) for a list of examples. 
\usepackage{thmtools}
\usepackage[framemethod=TikZ]{mdframed}
	\mdfdefinestyle{mdrecbox}
		{%
			linewidth=0.5pt,
			skipabove=12pt,
			frametitleaboveskip=5pt,
			frametitlebelowskip=0pt,
			skipbelow=2pt,
			frametitlefont=\bfseries,
			innertopmargin=4pt,
			innerbottommargin=8pt,
			nobreak=true,
		}
	\declaretheoremstyle
		[
			headfont=\bfseries,
			mdframed={style=mdrecbox},
			headpunct={\\[3pt]},
			postheadspace={0pt},
		]
		{thmrecbox}
	\declaretheorem[style=thmrecbox,name=Example, numberlike=theorem]{examplebox}


\newenvironment{example}
	{
	\begin{examplebox} 
	\leavevmode
	\begin{enumerate}}
	{	\end{enumerate} 
	\end{examplebox}
	}


% Math blackboard font
\newcommand{\nc}{\newcommand}
\nc{\on}[1]{\operatorname{#1}}

\nc{\R}{\mathbb R}
\nc{\C}{\mathbb C}
\nc{\Q}{\mathbb Q}
\nc{\Z}{\mathbb Z}
\nc{\N}{\mathbb N}
\nc{\HH}{\mathbb H}
\nc{\DD}{\mathbb D}
\nc{\TT}{\mathbb T}

\nc{\cT}{\mathcal T}
\nc{\cP}{\mathcal P}
\nc{\cM}{\mathcal M}
\nc{\cC}{\mathcal C}
\nc{\cB}{\mathcal B}
\nc{\cG}{\mathcal G}
\nc{\cA}{\mathcal A}
\nc{\cS}{\mathcal S}
\nc{\cF}{\mathcal F}
\nc{\cL}{\mathcal L}
\nc{\cR}{\mathcal R}

\nc{\frakI}{\mathfrak I}

\nc{\diam}{\operatorname{diam}}     % diameter of a set
\nc{\osc}{\operatorname{osc}}       % oscillation of a function
\nc{\inter}{\mathrm{o}}             % interior of a set
\nc{\close}[1]{\overline{#1}}       % closure of a set
\nc{\supp}{\operatorname{supp}}     % support of a function

\nc{\Symp}{\mathsf{Sp}}
\nc{\SpOrthO}{\mathsf{SO(odd)}}
\nc{\SpOrthE}{\mathsf{SO(even)}}
\nc{\Orth}{\mathsf O}
\nc{\Unit}{\mathsf U}
\nc{\UnitSp}{\mathsf{USp}}

% Why would you ever use \epsilon
\renewcommand{\epsilon}{\varepsilon}

% Title: change problem set number as needed
\title
{
	\textsc{Determining optimal test functions for $2$-level densities}
} 

\author{Jason Zhao}
\date{\today}

\begin{document}

\begin{titlepage}
\maketitle

\begin{abstract}
	It is of great interest to study the order of vanishing at the central
point of $L$-functions, such as the famous Riemann zeta function and its
generalizations. For example, the Birch and Swinnerton-Dyer conjecture
relates the order of vanishing for an elliptic curve $L$-function to the
rank of its group of rational solutions. Katz and Sarnak conjectured a
correspondence between the $n$-level density statistics of zeros from
families of $L$-functions (which essentially involves $n$-tuples of zeros)
with eigenvalues from random matrix ensembles, and in many cases that sums of smooth test functions, whose Fourier transforms are
finitely supported over scaled zeros in a family, converge to an integral of
the test function against a density $W_{n, G}$ depending on the
symmetry $G$ of the family (unitary, symplectic or orthogonal). This integral bounds the average order of vanishing at the central point of the corresponding family of $L$-functions. 

We can obtain better estimates on this vanishing in two ways. The first is
to do more number theory, and prove results for larger $n$ and greater
support; the second is to do functional analysis and obtain better test
functions to minimize the resulting integrals. We pursue the latter here
when $n=2$, minimizing  
	\[ \frac{1}{\Phi(0, 0)} \int_{{\mathbb R}^2} W_{2,G} (x, y) \Phi(x, y) dx dy \] 
over test functions $\Phi : {\mathbb R}^2
\to [0, \infty)$ with compactly supported Fourier transform. We study a
restricted version of this optimization problem, imposing that our test
functions take the form $\phi(x) \psi(y)$ for some fixed admissible $\psi(y)$ and
$\supp{\widehat \phi} \subseteq [-1, 1]$, thereby reducing the problem to
one analogous to the 1-level density case in optimizing over one-variable
test functions $\phi(x)$. Continuing with the analogy, Devlin and Miller
extended the functional analytic arguments of Iwaniec, Luo and Sarnak,
converting the restricted optimization problem to finding the unique $g
\in L^2[-1/2, 1/2]$ satisfying a Fredholm integral equation of the second
kind. From here we take two approaches. First, showing $g$ satisfies a homogeneous linear system
of differential equations, a method introduced by Freeman and Miller, from
which we can derive a closed form expression for $g$. Second, iterating to obtain a series representation of $g$ and truncating to compute explicit estimates on the minimum value.  We conclude by discussing improvements to previous bounds.
\end{abstract}

\tableofcontents
\end{titlepage}

\section{Prepatory material}
Denote by $G$ any one of the classical compact matrix groups, namely the orthogonal group $\Orth$, the even special orthogonal group $\SpOrthE$, the odd special orthogonal group $\SpOrthO$, and the symplectic group $\Symp$. The $1$-level densities of each of these groups are known to be
\begin{align*}
	W_\Orth (x) 
		&= 1 + \frac12 \delta_0 (x), \\
	W_{\SpOrthE} (x) 
		&= 1 + \frac{\sin 2\pi x}{2\pi x}, \\
	W_{\SpOrthO} (x)
		&= 1 - \frac{\sin 2 \pi x}{2 \pi x} + \delta_0 (x), \\
	W_{\Symp} (x)
		&= 1 - \frac{\sin 2\pi x}{2\pi x}.			
\end{align*}
Here $\delta_0 (x)$ is the Dirac delta centered at zero. The corresponding Fourier transforms take the form
	\[ \widehat W_G (\xi) = \delta_0 (\xi) + m_G (\xi) \]
where
\begin{align*}
	m_\Orth (\xi) 
		&= \frac12, \\
	m_{\SpOrthE} (\xi)
		&= \frac12 \mathbb 1_{[-1, 1]} (\xi), \\
	m_{\SpOrthO} (\xi)
		&= 1 - \frac12 \mathbb 1_{[-1, 1]} (\xi), \\
	m_{\Symp} (\xi)	
		&= - \frac12 \mathbb 1_{[-1, 1]} (\xi).	
\end{align*}
Define
	\[ \frakI_G (\sigma) = \inf_\phi \frac{1}{\phi(0)} \int_\R \phi(x) W_G (x) dx \]
where the infimum is taken over non-negative Schwarz functions $\phi : \R \to [0, \infty)$ with compactly supported Fourier transform satisfying $\supp (\widehat \phi) \subseteq [-2\sigma, 2\sigma]$. We refer to such $\phi$ as \textsc{test functions}. We are interested in studying the following:
\begin{enumerate}
	\item Finding test functions witnessing the infimum $\frakI_G (\sigma)$ for fixed $\sigma$ and group $G$, i.e. finding $\phi$ satisfying $\supp(\widehat \phi) \subseteq [-2\sigma, 2\sigma]$ and
			\[ \frakI_G (\sigma) = \frac{1}{\phi(0)} \int_\R \phi(x) W_G (x) dx \]
	 
	\item Computing the value of $\frakI_G (\sigma)$. 
	\item Analyzing the behavior of $\frakI_G$ as a real function on $(0, \infty)$. In particular, Freeman conjectured that $\frakI_G$ is continuous and real analytic except at integers and half-integers. 
\end{enumerate}
Fix $\sigma > 0$ and one of our aforementioned compact groups $G$; for brevity we suppress the subscripts $W := W_G$ and $m := m_G$. Moreover, whenever we define a new object in terms of $W$ and $m$, there is an implicit subscript $G$ to denote the dependence on the group. 

In the literature, no one tries to directly find the optimal test function $\phi$, instead appealing to functional analytic arguments. For example, it follows from a theorem of Ahiezer and the Paley-Wiener theorem that the optimal test function admits the form $\phi(z) = |h(z)|^2$, where $h : \C \to \C$ is an entire function of exponential type 1 and $h \in L^2 (\R)$. That is, its Fourier transform admits the form  
	\[ \widehat \phi (\xi) = (g * g^\smile) (\xi) \]
where $\supp g \subseteq [-\sigma, \sigma]$ and $g \in L^2 [-\sigma, \sigma]$ and $g^\smile (\xi) = \close{g(- \xi)}$. Notice that the Fourier transforms of the distributions $W$ are significantly easier to work with as they are step functions, so naturally we appeal to the Plancharel theorem to write
	\[\frac{1}{\phi(0)} \int_\R \phi(x) W (x) dx = \frac{1}{\int_\R \widehat \phi(\xi) d\xi} \int_\R \widehat \phi (\xi) \widehat W (\xi) d\xi . \]
Rewriting $\widehat \phi$ in terms of $g \in L^2 [-\sigma, \sigma]$ converts the problem to an equivalent optimization over the Hilbert space $L^2$ (for details of derivation, cf. Freeman Proposition 3.2), namely  
	\[ \frakI_G (\sigma) = \inf_{g \in L^2 [-\sigma, \sigma]} R(g), \]
where we define a quadratic form $R : L^2 [-\sigma, \sigma] \to \R$ and a self-adjoint operator $K: L^2 [-\sigma, \sigma] \to L^2 [-\sigma, \sigma]$ by 
	\[ R(g) = \frac{\langle (I + K)g,g \rangle}{| \langle g, 1 \rangle |^2}, \qquad (Kg) (x) = \int_{-\sigma}^\sigma m(x - y) g(y) dy, \]	
where $I$ is the identity operator on $L^2 [-\sigma, \sigma]$. By functional analysis hocus pocus, the constant function one is in the image of $I + K$. Moreover, there exists $f \in (\ker (I + K))^\perp$ satisfying the equation
	\[ (I + K) f \equiv 1. \]
This implies $A := \langle (I + K)f, f\rangle = \langle 1, f\rangle$ is a positive constant. ILS proved that in fact
	\[ \frakI_G (\sigma) = \inf_{g \in L^2 [-\sigma, \sigma]} R(g) = R(f) = \frac{1}{A}.  \]
ILS showed that if $\sigma = 1$, then the optimal functions $f \in L^2 [-1, 1]$ are even, and are trigonometric polynomials when restricted to $[0, 1]$. Freeman extended this result to arbitrary $\sigma > 0$. The brute force approach to this problem is to to find a trigonometric polynomial $f: [0, \sigma] \to \R$, i.e. taking the form
	\[ f(x) = \sum_{n \geq 0} a_n \cos(nx) + b_n \sin(nx) \]
where $a_n, b_n = 0$ for all $n \geq N$, such that $(I + K) f$ is a constant. Normalizing by this constant gives the desired equation $(I + K)f \equiv 1$. Since $f$ is even, we can extend it to a function on $[-\sigma, \sigma]$ to recover the desired optimal function. 
	


\section{One level}
The key observation in deriving the optimal $f \in L^2 [-\sigma, \sigma]$ is, as remarked earlier, it suffices to find $f$ such that $(I + K) f$ is constant. Differentiating gives us a delay diffferential equation which can be solved by hand. Here we work out the example of $\SpOrthE$ for $\sigma = 1$. We want to find $f \in L^2 [-1, 1]$ such that 
	\begin{equation}
		 f(x) + \frac12 \int_{-1}^1 f(y) \mathbb 1_{[-1. 1]} (x - y) dy = 1 	\label{eq:exampog}
	\end{equation} 
whenever $x \in [0, 1]$. We know $f$ is even, so symmetrizing allows us to reconstruct $f$ on $[-1, 0]$. First we rewrite the equation above in a more workable form; split the integral
	\[ \int_{-1}^1 f(y) \mathbb 1_{[-1. 1]} (x - y) dy = \int_{-1}^0 f(y) \mathbb 1_{[-1. 1]} (x - y) dy + \int_{0}^1 f(y) \mathbb 1_{[-1. 1]} (x - y) dy. \]
For the second integral on the right, $x, y \in [0, 1]$ so $x - y \in [-1, 1]$, so the characteristic function term is redundant. In the first integral on the left, $y \in [-1, 0]$ and $x \in [0, 1]$ implies $x - y \in [-1, 1]$ if and only if $y \in [x - 1, 0]$. By evenness, this is equivalent to integrating $f(y)$ for $y \in [0, 1 - x]$. Thus (\ref{eq:exampog}) is equivalent to 
	\begin{equation}
		f(x) + \frac12\int_0^1 f(y) dy + \frac12 \int_0^{1 - x} f(y) dy = 1		\label{eq:exampsimp}
	\end{equation}	
for $x \in [0, 1]$. We want to find $f$ such that the left hand side is constant with respect to $x$. Afterwards, setting $x = 1$ allows us to compute the desired normalizing constant. Differentiating (\ref{eq:exampsimp}) with respect to $x$ gives the delay differential equation
	\begin{equation}
		f' (x) - \frac12 f(1 - x) = 0						\label{eq:exampdelay}
	\end{equation}
for all $x \in [0, 1]$. Divine inspiration and a routine check using the identity $\sin (x) = \cos(x - \pi/2)$ shows that 
	\[ f(x) = \cos\left( \frac{x}{2} - \frac{\pi + 1}{4} \right) \]	
is a solution. Of course, for those who are not divinely inspired, we can solve by differentiating (\ref{eq:exampdelay}) to obtain
	\begin{equation}
		f''(x) = - \frac12 f' (1 - x) = - \frac14 f(x). 
	\end{equation}
the second equality follows directly from (\ref{eq:exampdelay}). This is a standard linear differential equation with a two-parameter family of solutions given by $f(x) = a \cos(x/2) + b \sin(x/2)$. Substituting these solutions into (\ref{eq:exampdelay}) yields the linear system
	\[
		\begin{pmatrix}
			2 \cos(1/2) 	& 2 \sin(1/2) - 1 \\
			2 \sin(1/2) + 1	& - 2 \cos(1/2)
		\end{pmatrix}
		\begin{pmatrix}
			a \\
			b
		\end{pmatrix}
		= 
		\begin{pmatrix}
			0 \\
			0
		\end{pmatrix}
	\]
The matrix has determinant zero with non-trivial rank so the kernel of the matrix is a one-dimensional subspace, i.e. all solutions to (\ref{eq:exampdelay}) are scalar multiples of our divinely inspired answer. 
	

\section{Quadratic kernel}
There are three key observations. First, the kernels take the form of quadratic polynomials in $|x|$ on the interval $[-1, 1]$, i.e. $m_{G, \psi} (x) = -a - b |x| - c |x|^2$ whenever $x \in [-1, 1]$. Second, the solutions found in \cite{FreemanMiller} and \cite{ILS} to the $1$-level density case for varying support took the form of piecewise trigonometric polynomials, and in particular continuously differentiable everywhere except finitely many points. Third, the right-hand side of (\ref{eq:fredholm}) is constant with respect to $x \in [-1/2, 1/2]$. Therefore, assuming $g$ is sufficiently smooth, we can differentiate the Fredholm integral equation (\ref{eq:fredholm}) to obtain a corresponding system of linear homogeneous differential equations. 

\begin{lemma}
	If $g \in L^2 [-1/2, 1/2]$ is smooth and solves
		\begin{equation}
			1 = g(x) + \int_{-1/2}^{1/2} (a + b |x - y| + c |x - y|^2) g(y) dy, 	\label{eq:fredholmquad}
		\end{equation}
	then it satisfies for $x \in [-1/2, 1/2]$ the following system of equations,  
	\begin{align}
		1
			&= g(0) + a \int_{-1/2}^{1/2} |y| g(y) dy, \label{eq:diff1}\\
		0 
			&= g'' (x) + 2b g(x) + 2c \int_{-1/2}^{1/2} g(y) dy, \label{eq:diff2} \\
		0
			&= g''' (x) + 2b g'(x).		\label{eq:diff3}	
	\end{align}
\end{lemma}

\begin{proof}
	Notice that 
	\[  \int_{-1/2}^{1/2} |x - y| g(y) dy = \int_{-1/2}^x (x - y) g(y) dy + \int_x^{1/2} (y - x) g(y) dy. \]
Differentating under the integral sign gives	
	\begin{align*}
		\frac{d}{dx}  \int_{-1/2}^{1/2} |x - y| g(y) dy 
			&= \int_{-1/2}^x g(y) dy - \int_x^{1/2} g(y) dy, \\
		\frac{d^2}{dx^2}  \int_{-1/2}^{1/2} |x - y| g(y) dy 
			&= 2g(x)		
	\end{align*}
	and
	\begin{align*}
		\frac{d}{dx} \int_{-1/2}^{1/2} (x - y)^2 g(y) dy
			&= \int_{-1/2}^{1/2} (2x - 2y) g(y) dy, \\
		\frac{d^2}{dx^2} \int_{-1/2}^{1/2} (x - y)^2 g(y) dy
			&= 2 \int_{-1/2}^{1/2} g(y) dy.
	\end{align*}
	And on the First Day of Genesis, He the God of Math deemed the rest to be trivial. 
\end{proof}

\begin{theorem}
	The solutions take the form
		\[ g(x) = \frac{6b^{3/2} (b + c) \cos(\sqrt{2b} x) - 6 \sqrt{2} b c \sin{\sqrt{b/2}}}{6 \sqrt{b} (b + c)^2 \cos{\sqrt{b/2}} + \sqrt{2} (6 a b^2 + 3b^3 + 3b^2 c + b c (c - 12) - 6c^2) \sin (\sqrt{b/2})}. \]	
\end{theorem}

\begin{proof}
	Assuming $g$ is even, has solutions of (\ref{eq:diff3}) take the form $A \cos(\sqrt{2b} x) + C$. This has two degrees of freedom. Substituting into $(\ref{eq:diff2})$ reduces to one degree of freedom, 
	\begin{align*}
		0 
			&= g'' (x) + 2b g(x) + 2c \int_{-1/2}^{1/2} g(y) dy \\
			&= 2C (b + c) + \frac{4A c}{\sqrt{2b}} \sin\left( \frac{\sqrt{2b}}{2} \right).
	\end{align*}
This shows that $g$ is a scalar multiple of  
	\[ \cos(\sqrt{2b} x) - \frac{2c}{b + c} \frac{\sin\left( \sqrt{b/2} \right)}{\sqrt{2b}}. \]	
Obviously this implies that $b \geq 0$ if we want to avoid nasty complexifications. Pumping into Wolfram Alpha, we obtain
	\[ g(x) = \frac{6b^{3/2} (b + c) \cos(\sqrt{2b} x) - 6 \sqrt{2} b c \sin{\sqrt{b/2}}}{6 \sqrt{b} (b + c)^2 \cos{\sqrt{b/2}} + \sqrt{2} (6 a b^2 + 3b^3 + 3b^2 c + b c (c - 12) - 6c^2) \sin (\sqrt{b/2})}. \]	
\end{proof}	


\section{Worked example: two-level $\SpOrthE$ for $\sigma = 1/2$}
For the two level density, we are interested in finding an optimal test function $\Phi : \R^2 \to \R$ with Fourier transform supported in $[-2\sigma, -2\sigma] \times [-2\sigma, -2\sigma]$ minimizing 
	\[ \frac{1}{\Phi(0, 0)} \int_{\R^2} \Phi(x, y) W_G (x, y) dx dy.  \]
Putting restrictions on $\Phi$, such as assuming it is of the form $\Phi(x, y) = \phi(x) \psi (y)$, makes the problem much easier while hopefully giving us a good upper bound on the true value of the minimum. Charlie was interested in $\sigma = 1/2$ with 
	\[\psi (x) = \left(\frac{\sin(\pi x)}{\pi x}\right)^2, \qquad \widehat \psi (x) = (1 - |x|) \mathbb 1_{[-1, 1]} (x). \] 
Analogous to the set-up in Appendix A of ILS and Jesse Freeman's thesis, Charlie found that finding the optimal $\phi: \R \to \R$ was equivalent to finding the unique solution $g \in L^2 [-1/2, 1/2]$ satisfying equation
\begin{equation}
	g(x) - \int_{-1/2}^{1/2} m_G (x - y) g(y) dy = 1,		\label{eq:fredholm}
\end{equation}
for all $x \in [-1/2, 1/2]$. Charlie proved that $g$ is even, increasing on $[-1/2, 0]$ and decreasing on $[0, 1/2]$. This seems to suggest that solutions are of the form $g (x) = a \cos (p x) + c$ for some period $p > 0$ and constants $a, c \in \R$. We show that this is indeed the case for $\SpOrthE$, which has corresponding kernel
\begin{equation}
	m_{\SpOrthE} (x) = -\frac12 + \frac43 (1 - |x|)^2 + \frac23 (1 - x^2).	\label{eq:mSpOrthE}
\end{equation}
Substituting (\ref{eq:mSpOrthE}) into (\ref{eq:fredholm}) gives
\[
	- \frac32 \int_{-1/2}^{1/2} g(y) dy + \frac83  \int_{-1/2}^{1/2} |x - y| g(y)dy - \frac23  \int_{-1/2}^{1/2} (x - y)^2 g(y) dy = 1
\]
for all $x \in [-1/2, 1/2]$. Notice that 
	\[  \int_{-1/2}^{1/2} |x - y| g(y) dy = \int_{-1/2}^x (x - y) g(y) dy + \int_x^{1/2} (y - x) g(y) dy. \]
Differentating under the integral sign gives	
\begin{align*}
	\frac{d}{dx}  \int_{-1/2}^{1/2} |x - y| g(y) dy 
		&= \int_{-1/2}^x g(y) dy - \int_x^{1/2} g(y) dy, \\
	\frac{d^2}{dx^2}  \int_{-1/2}^{1/2} |x - y| g(y) dy 
		&= 2g(x)		
\end{align*}
and
\begin{align*}
	\frac{d}{dx} \int_{-1/2}^{1/2} (x - y)^2 g(y) dy
		&= \int_{-1/2}^{1/2} (2x - 2y) g(y) dy, \\
	\frac{d^2}{dx^2} \int_{-1/2}^{1/2} (x - y)^2 g(y) dy
		&= 2 \int_{-1/2}^{1/2} g(y) dy.
\end{align*}
Thus, (\ref{eq:fredholm}) becomes, after differentiating three times with respect to $x$, 
	\[ g'''(x) + \frac{16}{3} g'(x) = 0 \]
for all $x \in [-1/2, 1/2]$. The solution set consists of the trigonometric polynomials $a \cos(4x/\sqrt 3) + b \sin(4x/\sqrt 3) + c$ for some constants $a, b, c \in \R$. We know $g$ is even, so this forces $b = 0$. To obtain the remaining coefficients, we differentiate (\ref{eq:fredholm}) twice to obtain
	\begin{align*}
		0 
			&= g'' (x) + \frac{16}{3} g(x) - \frac43 \int_{-1/2}^{1/2} g(y) dy \\
			&= 4 c - \frac43  a \int_{-1/2}^{1/2} \cos(4y/\sqrt 3) dy.
	\end{align*}
Rearranging, 
	\[ c = \frac13 a \int_{-1/2}^{1/2} \cos(4y/\sqrt 3) dy = \frac23 a \int_{0}^{1/2} \cos(4y/\sqrt 3) dy = \frac{\sin(2/\sqrt 3)}{2 \sqrt 3} a. \]	
This shows that $g$ is a scalar multiple of $\cos(4x/\sqrt 3) + \frac{\sin(2/\sqrt 3)}{2 \sqrt 3}$. Plugging into (\ref{eq: fredholm}) for $x = 0$ should give the desired multiplicative constant. 

\begin{theorem}
	The solutions to the integral equation
		\[ g_G (x) - \int_{-1/2}^{1/2} m_G (x - y) g_G (y) dy = 1 \]
	for $x \in [-1/2, 1/2]$ are 
		\begin{align*}
			g_{\SpOrthE} (x) 
				&= \frac{216 \cos(4x/\sqrt 3) + 36 \sqrt 3 \sin(2/\sqrt 3)}{-162 \cos(2/\sqrt 3) + 5 \sqrt{3} \sin(2/\sqrt 3}), \\
			g_{\SpOrthO} (x)
				&= \frac{8 \cos (4x) + 12 \sin(2)}{2 \cos (2) + 3 \sin(2)}, \\
			g_{\Unit} (x)
				&= \frac{6 \cos (2x) + 6 \sin(1)}{3 \cos(1) + 4 \sin(1)}, \\
			g_{\Orth} (x)
				&= \frac{6 \cos(2x \sqrt 2) + 3 \sqrt 2 \sin(\sqrt 2)}{3 \cos(\sqrt 2) + \sqrt 2 \sin (\sqrt 2)}			
		\end{align*}	
\end{theorem}

\section{Iteration}
We have solved the minimization problem for fixed $\psi$; our goal now is to generalize by optimizing over a larger class of $\psi$. Supposing that we can write $\widehat\psi = g * g$ for some even real valued function $g \in L^2 [-1/2, 1/2]$, we consider $g$ which are solutions to the earlier minimization problem, namely functions of the form 
	\[ g(x) = a \cos (bx) + c. \]
Since the problem is scaling invariant, we can normalize the constant term $c = 1$. For brevity, denote $h : \R \to \R$ the Fourier inverse of the map $x \mapsto \cos (bx) \mathbb 1_{[-1/2, 1/2]} (x)$. WolframAlpha gives us explicitly
	\begin{equation}
		h(x) = \int_{-1/2}^{1/2} e^{2\pi i x y } \cos(by) dy = \frac{2b \sin(b/2) \cos(\pi x) - 4\pi x \cos(b/2) \sin(\pi x)}{b^2 - 4 \pi^2 x^2}. 
	\end{equation}	
For notational convenience, denote $\cF$ the Fourier transform operator, i.e. $\cF \phi = \widehat \phi$. Viewing $g$ as a function on $\R$ supported on $[-1/2, 1/2]$, its Fourier inverse takes the form
	\[ 
		\cF^{-1} (g) (x) = a \int_{-1/2}^{1/2} e^{2\pi i x y} \cos(by) dy + \int_{-1/2}^{1/2} e^{2\pi i x y} dy  = a h(x) + \frac{\sin(\pi x)}{\pi x}.
	\]
As the Fourier transform and its inverse take convolutions to products and vice versa, we have 
	\begin{equation}
		\psi(x) 
			= \cF^{-1 } {g * g} (x)
			= \left[\cF^{-1}(g) (x)\right]^2 
			= [ah(x)]^2 + 2 a h(x) \frac{\sin(\pi x)}{\pi x} + \left( \frac{\sin (\pi x)}{\pi x} \right)^2.
	\end{equation}
Notice the last term is exactly our original $\psi$ that was used earlier. This is unsurprising since we are pushing around convolutions and Fourier transforms of trigonometric polynomials, so by linearity the constant term should pop out. 
	\[ \widehat \psi (x) = a \cF \left( [h(x)]^2\right)+  a \cF { \left( h(x) \frac{\sin (\pi x)}{\pi x} \right)} + (1 - |x|) \mathbb 1_{[-1, 1]} (x). \]	

\begin{thebibliography}{9999}

%number of characters controls spacing; must be longer than longest name

\bibitem[ILS]{ILS}
\newblock H. Davenport, \emph{Multiplicative Number Theory}, second
edition, Graduate Texts in Mathematics \textbf{74},
Springer-Verlag, New York, 1980, revised by H. Montgomery.

\bibitem[FM]{FreemanMiller}
\newblock A. Y. Khinchin, \emph{Continued Fractions},  Third
Edition, The University of Chicago Press, Chicago 1964.


\end{thebibliography}

\end{document}