The key observation in deriving the optimal $f \in L^2 [-\sigma, \sigma]$ is, as remarked earlier, it suffices to find $f$ such that $(I + K) f$ is constant. Differentiating gives us a delay diffferential equation which can be solved by hand. Here we work out the example of $\SpOrthE$ for $\sigma = 1$. We want to find $f \in L^2 [-1, 1]$ such that 
	\begin{equation}
		 f(x) + \frac12 \int_{-1}^1 f(y) \mathbb 1_{[-1. 1]} (x - y) dy = 1 	\label{eq:exampog}
	\end{equation} 
whenever $x \in [0, 1]$. We know $f$ is even, so symmetrizing allows us to reconstruct $f$ on $[-1, 0]$. First we rewrite the equation above in a more workable form; split the integral
	\[ \int_{-1}^1 f(y) \mathbb 1_{[-1. 1]} (x - y) dy = \int_{-1}^0 f(y) \mathbb 1_{[-1. 1]} (x - y) dy + \int_{0}^1 f(y) \mathbb 1_{[-1. 1]} (x - y) dy. \]
For the second integral on the right, $x, y \in [0, 1]$ so $x - y \in [-1, 1]$, so the characteristic function term is redundant. In the first integral on the left, $y \in [-1, 0]$ and $x \in [0, 1]$ implies $x - y \in [-1, 1]$ if and only if $y \in [x - 1, 0]$. By evenness, this is equivalent to integrating $f(y)$ for $y \in [0, 1 - x]$. Thus (\ref{eq:exampog}) is equivalent to 
	\begin{equation}
		f(x) + \frac12\int_0^1 f(y) dy + \frac12 \int_0^{1 - x} f(y) dy = 1		\label{eq:exampsimp}
	\end{equation}	
for $x \in [0, 1]$. We want to find $f$ such that the left hand side is constant with respect to $x$. Afterwards, setting $x = 1$ allows us to compute the desired normalizing constant. Differentiating (\ref{eq:exampsimp}) with respect to $x$ gives the delay differential equation
	\begin{equation}
		f' (x) - \frac12 f(1 - x) = 0						\label{eq:exampdelay}
	\end{equation}
for all $x \in [0, 1]$. Divine inspiration and a routine check using the identity $\sin (x) = \cos(x - \pi/2)$ shows that 
	\[ f(x) = \cos\left( \frac{x}{2} - \frac{\pi + 1}{4} \right) \]	
is a solution. Of course, for those who are not divinely inspired, we can solve by differentiating (\ref{eq:exampdelay}) to obtain
	\begin{equation}
		f''(x) = - \frac12 f' (1 - x) = - \frac14 f(x). 
	\end{equation}
the second equality follows directly from (\ref{eq:exampdelay}). This is a standard linear differential equation with a two-parameter family of solutions given by $f(x) = a \cos(x/2) + b \sin(x/2)$. Substituting these solutions into (\ref{eq:exampdelay}) yields the linear system
	\[
		\begin{pmatrix}
			2 \cos(1/2) 	& 2 \sin(1/2) - 1 \\
			2 \sin(1/2) + 1	& - 2 \cos(1/2)
		\end{pmatrix}
		\begin{pmatrix}
			a \\
			b
		\end{pmatrix}
		= 
		\begin{pmatrix}
			0 \\
			0
		\end{pmatrix}
	\]
The matrix has determinant zero with non-trivial rank so the kernel of the matrix is a one-dimensional subspace, i.e. all solutions to (\ref{eq:exampdelay}) are scalar multiples of our divinely inspired answer. 
	