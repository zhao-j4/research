\documentclass[11pt, reqno]{article}

\usepackage{amsfonts,latexsym,amsthm,amssymb,amsmath,amscd,euscript,bm}
\usepackage[sc]{mathpazo}
\usepackage[margin = 2cm]{geometry}
\usepackage{enumitem}
\usepackage{hyperref}
% sets numbering of enumerate to a, b, c, ...
\renewcommand{\theenumi}{\alph{enumi}}

% Theorems, propositions, etc.
\newtheorem{theorem}{Theorem}
\newtheorem{proposition}[theorem]{Proposition}
\newtheorem{lemma}[theorem]{Lemma}
\newtheorem{corollary}[theorem]{Corollary}

\theoremstyle{definition}
\newtheorem{definition}[theorem]{Definition}
\newtheorem*{claim}{Claim}

\theoremstyle{remark}
\newtheorem*{remark}{Remark}
\newtheorem*{notation}{Notation}

% Allows us of widecheck since loading mathabx causes problems 
\DeclareFontFamily{U}{mathx}{\hyphenchar\font45}
\DeclareFontShape{U}{mathx}{m}{n}{
      <5> <6> <7> <8> <9> <10>
      <10.95> <12> <14.4> <17.28> <20.74> <24.88>
      mathx10
      }{}
\DeclareSymbolFont{mathx}{U}{mathx}{m}{n}
\DeclareFontSubstitution{U}{mathx}{m}{n}
\DeclareMathAccent{\widecheck}{0}{mathx}{"71}
\DeclareMathAccent{\wideparen}{0}{mathx}{"75}

% Examples put into boxes, just an aesthetic choice
% {examplebox} for a single example, {example} is a list environment (\item) for a list of examples. 
\usepackage{thmtools}
\usepackage[framemethod=TikZ]{mdframed}
	\mdfdefinestyle{mdrecbox}
		{%
			linewidth=0.5pt,
			skipabove=12pt,
			frametitleaboveskip=5pt,
			frametitlebelowskip=0pt,
			skipbelow=2pt,
			frametitlefont=\bfseries,
			innertopmargin=4pt,
			innerbottommargin=8pt,
			nobreak=true,
		}
	\declaretheoremstyle
		[
			headfont=\bfseries,
			mdframed={style=mdrecbox},
			headpunct={\\[3pt]},
			postheadspace={0pt},
		]
		{thmrecbox}
	\declaretheorem[style=thmrecbox,name=Example, numberlike=theorem]{examplebox}


\newenvironment{example}
	{
	\begin{examplebox} 
	\leavevmode
	\begin{enumerate}}
	{	\end{enumerate} 
	\end{examplebox}
	}


% Math blackboard font
\newcommand{\nc}{\newcommand}
\nc{\on}[1]{\operatorname{#1}}

\nc{\R}{\mathbb R}
\nc{\C}{\mathbb C}
\nc{\Q}{\mathbb Q}
\nc{\Z}{\mathbb Z}
\nc{\N}{\mathbb N}
\nc{\HH}{\mathbb H}
\nc{\DD}{\mathbb D}
\nc{\TT}{\mathbb T}

\nc{\cT}{\mathcal T}
\nc{\cP}{\mathcal P}
\nc{\cM}{\mathcal M}
\nc{\cC}{\mathcal C}
\nc{\cB}{\mathcal B}
\nc{\cG}{\mathcal G}
\nc{\cA}{\mathcal A}
\nc{\cS}{\mathcal S}
\nc{\cF}{\mathcal F}
\nc{\cL}{\mathcal L}
\nc{\cR}{\mathcal R}

\nc{\frakI}{\mathfrak I}

\nc{\diam}{\operatorname{diam}}     % diameter of a set
\nc{\osc}{\operatorname{osc}}       % oscillation of a function
\nc{\inter}{\mathrm{o}}             % interior of a set
\nc{\close}[1]{\overline{#1}}       % closure of a set
\nc{\supp}{\operatorname{supp}}     % support of a function

\nc{\Symp}{\mathsf{Sp}}
\nc{\SpOrthO}{\mathsf{SO(odd)}}
\nc{\SpOrthE}{\mathsf{SO(even)}}
\nc{\Orth}{\mathsf O}
\nc{\Unit}{\mathsf U}
\nc{\UnitSp}{\mathsf{USp}}

% Why would you ever use \epsilon
\renewcommand{\epsilon}{\varepsilon}

% Title: change problem set number as needed
\title
{
	\textsc{Fourier analytic approach to deriving the optimal $g$}
} 

\author{Jason Zhao}
\date{\today}

\begin{document}

\maketitle

Given a continuous kernel $m : [-1, 1] \to \R$, we want to find $g \in L^2 [-1/2, 1/2]$ satisfying 
	\[ g(x) - \int_{-1/2}^{1/2} m(x - y) g(y) dy = 1 \]
for $x \in [-1/2, 1/2]$. Writing $g$ as the sum of iterates of $m$ is one method, however, computing the resulting integral of $g$ up to $n$ terms involves computing along the order of $1 + 2 + \dots + n$ many integrals. Very computationally inefficient. What if we instead computed the Fourier coefficients instead? Viewing $g$ and $m$ as 2-periodic functions, i.e. extending $g$ by zero on $[-1,1]$ and then $g$ and $m$ extended 2-periodically to $\R$, the integral equation becomes
	\[ g(x) - (m * g) (x) = \begin{cases} 1, & \text{if } |x| \leq 1/2, \\ -(m * g)(x), &\text{if } 1/2 \leq |x| \leq 1. \end{cases}.  \]
Let $a(n), b(n), c(n), d(n)$ be the Fourier coefficients of $g, m, {\mathbb 1}_{[-1/2, 1/2]}, {\mathbb 1}_{[-1, -1/2] \cup [1/2, 1]}$, then since Fourier transforms take convolutions to products and vice versa, we obtain
	\[ a(n) (1 - b(n)) = c(n) - \sum_{k \in \Z} a(n - k) b(n - k) d(k). \]
This gives an infinite ``linear'' system of equations to solve for $a(n)$. If we can somehow hocus pocus that without too much computational difficult, then computing an approximation for the integral of $g$ up to $n$-terms in the Fourier expansion only involves $n$ or so integrals. Accounting for the convergence rate of the Fourier coefficients we can estimate how close this truncation is to the actual integral. 		 

\end{document}