% user docs for compositio.cls
\def\Filedate{2002/12/18} % date this file last revised
\def\Fileversion{1.0b}    % version of compositio.cls documented

\documentclass{compositio}
% If you have the AMSLaTeX distribution installed on your system,
% please delete the "[noams]" option above.

% definitions specific to this author guide only

\newcommand{\twovec}[2]
{\left(\begin{array}{c}
                        #1    \\
                        #2
                          \end{array}\right) }

\newcommand{\mattwo}[4]
{\left(\begin{array}{cc}
                        #1  & #2   \\
                        #3 &  #4
                          \end{array}\right) }

\newcommand{\mattwon}[4]
{\begin{array}{cc}
                        #1  & #2   \\
                        #3 &  #4
                          \end{array} }


\newcommand{\matthree}[9]
{\left(\begin{array}{ccc}
                        #1  & #2 & #3  \\
                        #4 &  #5 & #6 \\
                        #7 &  #8 & #9
                          \end{array}\right) }

\newcommand{\dettwo}[4]
{\left|\begin{array}{cc}
                        #1  & #2   \\
                        #3 &  #4
                          \end{array}\right| }

\newcommand{\detthree}[9]
{\left|\begin{array}{ccc}
                        #1  & #2 & #3  \\
                        #4 &  #5 & #6 \\
                        #7 &  #8 & #9
                          \end{array}\right| }



\newcommand*{\code}[1]{{\mdseries\texttt{#1}}}
\newcommand*{\pkg}[1]{{\mdseries\textsf{#1}}}
\renewcommand{\topfraction}{0.9}
\renewcommand{\contentsname}{Contents\\{\footnotesize\normalfont(A table
of contents should normally not be included)}}
%

\newcommand\st{\mbox{s.t.\ }}
\newcommand\be{\begin{equation}}
\newcommand\ee{\end{equation}}
\newcommand\bp{\begin{proof}}
\newcommand\ep{\end{proof}}

\newcommand\bea{\begin{eqnarray}}
\newcommand\eea{\end{eqnarray}}
\newcommand\bi{\begin{itemize}}
\newcommand\ei{\end{itemize}}
\newcommand\ben{\begin{enumerate}}
\newcommand\een{\end{enumerate}}
\newcommand\bc{\begin{center}}
\newcommand\ec{\end{center}}
\newcommand\ba{\begin{array}}
\newcommand\ea{\end{array}}

\newtheorem{thm}{Theorem}[section]
\newtheorem{conj}[thm]{Conjecture}
\newtheorem{cor}[thm]{Corollary}
\newtheorem{lem}[thm]{Lemma}
\newtheorem{prop}[thm]{Proposition}
\newtheorem{exa}[thm]{Example}
\newtheorem{defi}[thm]{Definition}
\newtheorem{exe}{Problem}
\newtheorem{rek}[thm]{Remark}


\def\notdiv{\ \mathbin{\mkern-8mu|\!\!\!\smallsetminus}}
\newcommand{\Qoft}{\Bbb{Q}(t)}  %use in linux
\newcommand{\done}{\Box} %use in linux
\newcommand{\R}{\ensuremath{\Bbb{R}}}
\newcommand{\C}{\ensuremath{\Bbb{C}}}
\newcommand{\Z}{\ensuremath{\Bbb{Z}}}
\newcommand{\Q}{\Bbb{Q}}

\newcommand{\ga}{\alpha}     %lowercase alpha
\newcommand{\gb}{\beta}      %lowercase beta
\newcommand{\gd}{\delta}     %lowercase delta
\newcommand{\gep}{\epsilon}   %lowercase epsilon
\newcommand{\G}{\Gamma}     %Capital Gamma
\newcommand{\g}{\gamma}     %lowercase gamma
\newcommand{\gL}{\Lambda}    %capital lamda
\newcommand{\gl}{\lambda}

\newcommand{\lap}{\triangle}    %laplacian
\newcommand{\pr}{\partial}     %partial derivative
\newcommand{\hphi}{\widehat{\phi}}  %phi^
\newcommand{\phiint}{\int_{-\infty}^{\infty} \phi(y) dy}

\newcommand{\tp}{t_p}           %t_p
\newcommand{\pnddisc}{p \ {\notdiv} \ \lap}  %p not dividing disc
\newcommand{\pddisc}{p | \lap}          %p dividing disc
\newcommand{\lcp}{\frac{\log c_E}{2 \pi}}  %Log(c_E)/2Pi
\newcommand{\lnp}{\frac{\log N}{2 \pi}}     %Log(N)/2Pi

\newcommand{\epxxi}{e^{2 \pi i x \xi}}  %exp(2 Pi i x xi)
\newcommand{\enxxi}{e^{-2 \pi i x \xi}} %exp(-2Pi i x xi)
\newcommand{\epzxi}{e^{2 \pi i (x+iy) \xi}}
\newcommand{\enzxi}{e^{-2 \pi i (x+iy) \xi}}

\newcommand{\fof}{\frac{1}{4}}  %oneforth
\newcommand{\foh}{\frac{1}{2}}  %onehalf
\newcommand{\fot}{\frac{1}{3}}  %onethird
\newcommand{\fop}{\frac{1}{\pi}}    %1/pi
\newcommand{\ftp}{\frac{2}{\pi}}    %2/pi
\newcommand{\fotp}{\frac{1}{2 \pi}} %1/2pi
\newcommand{\fotpi}{\frac{1}{2 \pi i}}
\newcommand{\fologn}{O\Big(\frac{1}{\log N}\Big)}
\newcommand{\fllnln}{O\Big(\frac{\log \log N}{\log N}\Big)}
\newcommand{\flogpn}{\frac{\log p}{\log N}}


\newcommand{\FD}{\mathcal{F}} %fundamental domain
\newcommand{\HP}{\ensuremath{\mathbf{H}}} %upperhalf plane
\newcommand{\js}[1]{ { \underline{#1} \choose p} }
\newcommand{\zsum}[1]{ \sum_{#1 = 0}^{p-1} }
\newcommand{\osum}[1]{ \sum_{#1 = 1}^{p-1} }

%additional commands for 2-density

\newcommand{\oof}{\frac{1}{|\mathcal{F}|}} % 1/|F|
\newcommand{\oop}{\frac{1}{p}}
\newcommand{\oopi}{\frac{1}{p_i}}
\newcommand{\oopo}{\frac{1}{p_1}}
\newcommand{\oopt}{\frac{1}{p_2}}

\newcommand{\hf}{\widehat{f}}
\newcommand{\hfi}{\widehat{f_i}}
\newcommand{\hfo}{\widehat{f_1}}
\newcommand{\hft}{\widehat{f_2}}
\newcommand{\NF}{|\mathcal{F}|}   %|F|
\newcommand{\sumtinF}{\sum_{t \in \mathcal{F}}}
\newcommand{\ooNF}{\frac{1}{|\mathcal{F}|}}   %1/|F|
\newcommand{\sumef}{\sum_{E \in \mathcal{F}}}

\newcommand{\plogx}{ \frac{\log p}{\log X}}
\newcommand{\plogm}{\frac{\log p}{\log M}}
%\newcommand{\plognt}{\frac{\log p}{\log N_t}}
% change in notation: use C(t) for conductor, not C(t)
\newcommand{\plogCt}{\frac{\log p}{\log C(t)}}
\newcommand{\pilogm}{\frac{\log p_i}{\log M}}
\newcommand{\pjlogm}{\frac{\log p_j}{\log M}}
\newcommand{\pologm}{\frac{\log p_1}{\log M}}
\newcommand{\ptlogm}{\frac{\log p_2}{\log M}}

\newcommand{\jsi}[1]{ { \underline{#1} \choose p_i} }
\newcommand{\jso}[1]{ { \underline{#1} \choose p_1} }
\newcommand{\jst}[1]{ { \underline{#1} \choose p_2} }
\newcommand{\jsthree}[1]{ { \underline{#1} \choose 3} }


\newcommand{\epi}[1]{ \mathbf{e}{ \underline{#1} \choose p_i} }
\newcommand{\epo}[1]{ \mathbf{e}{ \underline{#1} \choose p_1} }
\newcommand{\ept}[1]{ \mathbf{e}{ \underline{#1} \choose p_2} }
\newcommand{\epot}[1]{ \mathbf{e}{ \underline{#1} \choose p_1p_2} }
%\newcommand{\ep}[1]{ \mathbf{e}{ \underline{#1} \choose p} }
\newcommand{\eP}[1]{ \mathbf{e}{ \underline{#1} \choose P} }

%  **********************************************
%  NEW COMMANDS FOR n-LEVEL DENSITIES
%  **********************************************

\newcommand{\nF}{|\mathcal{F}|}
\newcommand{\uF}{\underline{F}}
\newcommand{\nuF}{\nu(\underline{F})}

\newcommand{\jsj}[1]{ { \underline{#1} \choose p_j} }
\newcommand{\jsk}[1]{ { \underline{#1} \choose p_k} }
\newcommand{\epj}[1]{ \mathbf{e}{ \underline{#1} \choose p_j} }
\newcommand{\epk}[1]{ \mathbf{e}{ \underline{#1} \choose p_k} }

\newcommand{\jsqk}[1]{ { \underline{#1} \choose q_k} }
\newcommand{\epqk}[1]{ \mathbf{e}{ \underline{#1} \choose q_k} }
\newcommand{\eppq}[1]{ \mathbf{e}{ \underline{#1} \choose PQ} }
\newcommand{\epq}[1]{ \mathbf{e}{ \underline{#1} \choose PQ} }
\newcommand{\eQ}[1]{ \mathbf{e}{ \underline{#1} \choose Q} }
\newcommand{\epqt}[1]{ \mathbf{e}{ \Bigg(\frac{{#1}}{PQ}t\Bigg)}  }
\newcommand{\eppqa}[1]{ \mathbf{e}{ \Bigg(\frac{{#1}}{PQ}a\Bigg)}  }
\newcommand{\eppqb}[1]{ \mathbf{e}{ \Bigg(\frac{{#1}}{PQ}b\Bigg)}  }

\newcommand{\epa}[1]{ \mathbf{e}{ \underline{#1} \choose P_AQ} }
\newcommand{\epaa}[1]{ \mathbf{e}{ \Bigg(\frac{{#1}}{P_AQ}a\Bigg)}  }
\newcommand{\epab}[1]{ \mathbf{e}{ \Bigg(\frac{{#1}}{P_AQ}b\Bigg)}  }
\newcommand{\logpne}{\frac{\log p}{\log N_E}}
\newcommand{\loglognn}{\frac{\log \log N_E}{\log N_E}}

\newcommand{\plogne}{\frac{\log p}{\log N_E}}
\newcommand{\pologne}{\frac{\log p_1}{\log N_E}}
\newcommand{\ptlogne}{\frac{\log p_2}{\log N_E}}
\newcommand{\pilogne}{\frac{\log p_i}{\log N_E}}
\newcommand{\etam}{\frac{\eta_E}{\log M}}
\newcommand{\etameta}{\frac{\eta_E}{\log M + \eta_E}}
\newcommand{\oologm}{\frac{1}{\log M}}



%  **********************************************
%  NEW COMMANDS FOR n-LEVEL DENSITIES, 9t+1
%  **********************************************
\newcommand{\epaqb}[1]{ \mathbf{e}{ \underline{#1} \choose P_AQ_B} }
\newcommand{\epaqbt}[1]{ \mathbf{e}{ \Bigg(\frac{{#1}}{P_AQ_B}t\Bigg)}  }
\newcommand{\epqj}[1]{ \mathbf{e}{ \underline{#1} \choose q_j} }
\newcommand{\jsqj}[1]{ { \underline{#1} \choose q_j} }

%  **********************************************
%  NEW COMMANDS FOR Density Kernel Calculations
%  **********************************************

\newcommand{\chiint}{\chi_{[-\foh,\foh]}}
\newcommand{\kot}[1]{ \frac{\sin \pi({#1}) }{\pi ({#1})} }
\newcommand{\kkot}[1]{ \frac{\sin \pi {#1} }{\pi {#1} } }


%  **********************************************
%  NEW COMMANDS FOR DIRICHLET CHARACTERS SECTIONS
%  **********************************************

\newcommand{\logpm}{ \frac{\log p}{\log(m/\pi)} }      %logp / log(m/pi)
\newcommand{\logptm}{ 2\frac{\log p}{\log(m/\pi)} }    %2logp / log(m/pi)
\newcommand{\logpnp}{ \frac{\log p}{\log(\frac{N}{\pi})} }     %logp / log(N/pi)
\newcommand{\logptnp}{ 2\frac{\log p}{\log(\frac{N}{\pi})} }    %2logp / log(N/pi)
\newcommand{\bchi}{ \overline{\chi} }                               %Chi-Bar
\newcommand{\fologm}{O(\frac{1}{\log m})}                       %O(1/logm)
\newcommand{\fommt}{\frac{1}{m-2}}                             %1/m-2
\newcommand{\jsq}[1]{ { \underline{#1} \choose q} }            %(*/q)

%  **********************************************
%  NEW COMMANDS FOR N-LEVEL DENSITIES
%  **********************************************

\newcommand{\hkpn}{H_k^+(N)}
\newcommand{\hkmn}{H_k^-(N)}
\newcommand{\hkpmn}{H_k^\pm(N)}
\newcommand{\hkn}{H_k^\ast(N)}
\newcommand{\hksn}{H_k^\sigma(N)}

\newcommand{\hkpm}{H_k^+(M)}
\newcommand{\hkmm}{H_k^-(M)}
\newcommand{\hkpmm}{H_k^\pm(M)}
\newcommand{\hkm}{H_k^\ast(M)}

\newcommand{\jk}[2]{J_{k-1}\left( 4\pi \frac{ \sqrt{ #1 } }{ #2 }
\right) }
\newcommand{\sqn}{\sqrt{N}}

\newcommand{\phir}[1]{\widehat{\phi}\left( \frac{ \log p_{#1} }{\log R}
\right) }
\newcommand{\phirx}[1]{\widehat{\phi}\left( \frac{ \log x_{#1} }{\log R}
\right) }
\newcommand{\phiv}[1]{\widehat{\phi}\left( \frac{ \log #1 }{\log R}
\right) }
\newcommand{\hphiv}[1]{\widehat{\phi}\left( #1 \right) }
\newcommand{\flogr}[1]{\frac{ #1 }{\log R}}

\newcommand{\pfrac}[1]{\frac{2\log p_{#1}}{\sqrt{p_{#1}} \log R}}

\newcommand{\glp}[1]{\gl_f(p_{#1})}

\newcommand{\ils}{\cite{ILS}}
\newcommand\eq{{Equation\ }}
\newcommand{\chib}[1]{\overline{\chi}(#1)}
\newcommand{\chibt}[1]{\overline{\chi_2}(#1)}
\newcommand{\chibo}[1]{\overline{\chi_1}(#1)}
\newcommand{\chibthree}[1]{\overline{\chi_3}(#1)}
\newcommand{\db}{\overline{d}}
\newcommand{\mod}{\ mod \ }



    %=======================================================
    %   THIS IS WHERE YOU PUT SHORTCUT DEFINITIONS
    %========================================================


    %=======================================================
    %   THIS IS USED TO LABEL THE EQUATIONS 1.1, 1.2, ..., 2.1
    %========================================================
\renewcommand{\theequation}{\thesection.\arabic{equation}}



\begin{document}

\title{Calculating the Level Density a la Katz-Sarnak}
%
\author{Christopher P. Hughes}
\email{hughes@aimath.org}
\address{American Institute of Mathematics, Palo Alto, CA,
$94306$}
\author{Steven J. Miller}
\email{sjmiller@math.ohio-state.edu}
\address{Department of Mathematics, The Ohio State University, Columbus, Ohio}
%\address{American Institute of Mathematics, Palo Alto, CA, $94306$}
%\dedication{A dedication can be included here.}
\classification{11M26 (primary), 11G05, 11G40, 11M26 (secondary).
} \keywords{$2$-Level Density}
%\thanks{This file documents \pkg{compositio} version \Fileversion\ and
%was last revised \Filedate.}

\begin{abstract}
These are an expanded set of notes from \cite{Mil1}. Below we
calculate the 1 and 2-Level Densities for the Classical Compact
Groups, using the determinant expansions from Katz-Sarnak.
\end{abstract}

\maketitle

%\tableofcontents

%\vspace*{6pt}\tableofcontents  % for this guide only.
% A table of contents should normally not be included

%!!!!!!!!!!!!!!!!!!!!!!!!!!!!!!!!!!!!!!!new section!!!!!!!!!!!!!!!!!!!!!!!!!!!!!!!!
%!!!!!!!!!!!!!!!!!!!!!!!!!!!!!!!!!!!!!!!new section!!!!!!!!!!!!!!!!!!!!!!!!!!!!!!!!
%!!!!!!!!!!!!!!!!!!!!!!!!!!!!!!!!!!!!!!!new section!!!!!!!!!!!!!!!!!!!!!!!!!!!!!!!!


\section{$1$- and $2$-Level Density Kernels for the Classical Compact
Groups} \setcounter{equation}{0}

By \cite{KS1}, the $m$-level densities for the classical compact
groups are

\begin{eqnarray}\label{eqdensitykernels}
W_{m,\epsilon}(x) & = & \textbf{det} \Big(
K_\epsilon(x_i,x_j)\Big)_{i,j\leq m}  \nonumber\\ W_{m,O^+}(x) & =
& \textbf{det} (K_1(x_i,x_j))_{i,j\leq m} \nonumber\\ W_{m,O^-}(x)
& = & \textbf{det} (K_{-1}(x_i,x_j))_{i,j\leq m}  + \sum_{k=1}^m
\delta(x_k) \textbf{det}(K_{-1}(x_i,x_j))_{i,j\neq k} \nonumber\\
& = & (W_{m,O^-})_{1}(x) + (W_{m,O^-})_{2}(x) \nonumber\\
W_{m,O}(x) & = & \foh W_{m,O^+}(x) + \foh W_{m,O^-}(x) \nonumber\\
W_{m,U}(x) & = & \textbf{det} (K_0(x_i,x_j))_{i,j\leq m}
\nonumber\\ W_{m,Sp}(x) &=& \textbf{det}
(K_{-1}(x_i,x_j))_{i,j\leq m}
\end{eqnarray}

where $K(y) = \kkot{y}$, $ K_\epsilon(x,y) = K(x-y) + \epsilon
K(x+y)$ for $\epsilon = 0, \pm 1$, $O^+$ denotes the group
$SO(\mbox{even})$ and $O^-$ the group $SO(\mbox{odd})$.

In all applications below, there are enormous simplifications as
all functions are even (we do not need to worry about signs in the
Fourier transform or the inversion formulas).

\begin{defi}[$I(x)$] $I(x)$ will denote the characteristic
function of $[-1,1]$. \end{defi}

\subsection{Needed Fourier Transforms}

Let $\delta$ be the Dirac Delta functional: $\int f(x) \delta(x) =
f(0)$. Let $I(x) = \chi_{[-1,1]}(x)$ be the characteristic
function of the unit interval.

\begin{lem}$\widehat{1} = \delta$
\end{lem}

\bp This is proved in the theory of distributions. Formally, using
duality, one can argue $\int f \cdot 1 = \widehat{f}(0) = \int
\widehat{f} \cdot \delta$. \ep

\begin{lem}$\widehat{\ \ \chiint \ \ }(u) = K(u)$
\end{lem}

\bp
\begin{eqnarray}
\int_{-\infty}^{\infty} \chiint(x) e^{2\pi i xu} dx & = &
\int_{-\foh}^{\foh} \Big[\cos (2\pi xu) + i \sin (2 \pi xu) \Big]
dx \nonumber\\ & = & \int_{-\foh}^{\foh} \cos (2\pi xu) dx
\nonumber\\ & = & \frac{\sin \pi u}{\pi u}
\end{eqnarray} \ep

\begin{lem}$\widehat{K(2x)}(u) = \foh I(u)$.
\end{lem}

\bp
\begin{eqnarray}
\widehat{K(2x)}(u) & = & \int_{-\infty}^{\infty} K(2x) e^{2 \pi i
x u} dx \nonumber\\ & = & \int_{-\infty}^{\infty} K(2x) e^{2 \pi i
2x (\foh u)} \frac{2dx}{2} \nonumber\\ & = & \foh
\int_{-\infty}^{\infty} K(t) e^{2 \pi i t (\foh u)} dt \nonumber\\
& = & \foh \chiint\left(\foh u\right) = \foh I(u).
\end{eqnarray} \ep



\begin{lem}$\widehat{K^2}(u) = (1 - |u|)I(u)$.
\end{lem}

\bp We use duality for even functions: $\int f(x)g(x)dx = \int
\widehat{f}(y) \widehat{g}(y) dy$. See \cite{La2}, pages
$242-243$. Let $K_{u}(t) = K(t)e^{2\pi i u t}$. Then
$\widehat{K_{u}}(y) = \widehat{K}(y+u)$, and recall
$\widehat{K}(y) = \chiint(y)$. As $K$ is even, the arguments below
are justified.

\begin{eqnarray} \int_{-\infty}^\infty  K^2(t) e^{2 \pi i
ut} dt & = & \int_{-\infty}^\infty  \Big(K(t)\Big) \Big( K(t) e^{2
\pi i ut} \Big) dt \nonumber\\ & = & \int_{-\infty}^\infty K(t)
K_{u}(t) dt \nonumber\\ & = & \int_{-\infty}^\infty \chiint(y)
\chiint(y+u) dy
\end{eqnarray}

$\chiint(y) \chiint(y+u)$ is one on the intersection of $\{-\foh
\leq y \leq \foh\}$ and $\{-\foh \leq y + u \leq \foh\}$ and zero
elsewhere. If $|u|
> 1$, $\chiint(y) \chiint(y+u) = 0$, and the integral vanishes.
If $u \in [0,1]$, the intersection is $-\foh \leq y \leq \foh -
u$, and integrating over $y$ gives $1 - u$. If $u \in [-1,0]$, it
is one on the intersection of $\{-\foh \leq y \leq \foh\}$ and
$\{-\foh \leq y - |u| \leq \foh\}$. We get $-\foh + |u| \leq y
\leq \foh$, and integrating over $y$ gives $1 - |u|$. Therefore
the Fourier Transform of $K^2$ is $(1 - |u|)I(u)$. \ep



\subsection{$1$-Level Densities}

For $|u_1| \leq 1$, $\foh I(u_1) = -\foh I(u_1) + 1$.

\begin{eqnarray}
W_{1,O^+}(x_1) & = & \textbf{det} \Big( K_1(x_i,x_j) \Big)_{i,j
\leq 1} \nonumber\\ & = & K_1(x_1,x_1) = 1 + K(2x_1) \nonumber\\ &
= & 1(x_1) + K(2x_1) \nonumber\\ \widehat{W_{1,O^+} }(u_1) & = &
\delta(u_1) + \foh I(u_1).
\end{eqnarray}

\begin{eqnarray}
W_{1,O^-}(x_1) & = & \textbf{det} \Big( K_{-1}(x_i,x_j) \Big)_{i,j
\leq 1} \nonumber\\ & & + \sum_{k=1}^1 \delta(x_k) \textbf{det}
\Big( K_{-1}(x_i,x_j) \Big)_{i,j \neq 1} \nonumber\\ & = &
K_{-1}(x_1,x_1) + \delta(x_1) \nonumber\\ & = & 1 - K(2x_1) +
\delta(x_1) \nonumber\\ & = & 1(x_1) - K(2x_1) + \delta(x_1)
\nonumber\\ \widehat{W_{1,O^-} }(u_1) & = & \delta(u_1) - \foh
I(u_1) + 1(u_1).
\end{eqnarray}


\begin{eqnarray}
W_{1,Sp}(x_1) & = & \textbf{det} \Big( K_{-1}(x_i,x_j) \Big)
\nonumber\\ & = & K_{-1}(x_1,x_1) \nonumber\\ &=& 1(x_1) - K(2x_1)
\nonumber\\ \widehat{W_{1,Sp} }(u_1) & = & \delta(u_1) - \foh
I(u_1).
\end{eqnarray}


\begin{eqnarray}
W_{1,U}(x_1) & = & \textbf{det} \Big( K_0(x_i,x_j)\Big)
\nonumber\\ & = & K_0(x_1,x_1) = 1(x_1) \nonumber\\
\widehat{W_{1,U} }(u_1) & = & \delta(u_1).
\end{eqnarray}


We have shown
\begin{thm}[$1$-Level Densities]\label{thmonelevel}
\begin{eqnarray}
\widehat{W_{1,O^+} }(u) & = & \delta(u) + \foh I(u) \nonumber\\
\widehat{W_{1,O} }(u) & = & \delta(u) + \foh \nonumber\\
\widehat{W_{1,O^-} }(u) & = & \delta(u) - \foh I(u) + 1
\nonumber\\ \widehat{W_{1,Sp} }(u) & = & \delta(u) - \foh I(u)
\nonumber\\ \widehat{W_{1,U} }(u) & = & \delta(u).
\end{eqnarray}
For functions whose Fourier Transforms are supported in $[-1,1]$,
the three orthogonal densities are indistinguishable, though they
are distinguishable from $U$ and $Sp$. To detect differences
between the orthogonal groups using the $1$-level density, one
needs to work with functions whose Fourier Transforms are
supported beyond $[-1,1]$.
\end{thm}


\subsection{Preliminaries for the $2$-Level Densities}

\begin{eqnarray}
W_{2,\epsilon}(x) & = & \textbf{det} \Big( K_\epsilon(x_i,x_j)
\Big)_{i,j \leq 2} \nonumber\\ & = &
K_\epsilon(x_1,x_1)K_\epsilon(x_2,x_2) -
K_\epsilon(x_1,x_2)K_\epsilon(x_2,x_1) \nonumber\\ & = & \Big[ 1 +
\epsilon K(2x_1) \Big] \Big[1 + \epsilon K(2x_2) \Big] -
\nonumber\\ & & \Big[K(x_1-x_2) + \epsilon K(x_1+x_2) \Big] \Big[
K(x_2-x_1) + \epsilon K(x_1+x_2) \Big] \nonumber\\ & = &
W_{2,\epsilon,a}(x) - W_{2,\epsilon,b}(x).
\end{eqnarray}

We now calculate $\widehat{W_{2,\epsilon,a}}(u)$.

\begin{eqnarray}
W_{2,\epsilon,a}(x) & = & \Big[ 1 + \epsilon K(2x_1) \Big] \Big[1
+ \epsilon K(2x_2) \Big] \nonumber\\ & = & 1 + \epsilon K(2x_1) +
\epsilon K(2x_2) + \epsilon^2 K(2x_1) K(2x_2) \nonumber\\ & = &
1(x_1)1(x_2) + \epsilon K(2x_1) 1(x_2) + \epsilon 1(x_1) K(2x_2) +
\epsilon^2 K(2x_1) K(2x_2) \nonumber\\ \widehat{W_{2,\epsilon,
a}}(u) & = & \widehat{1(x_1)} \widehat{1(x_2)} + \epsilon
\widehat{K(2x_1)} \widehat{1(x_2)} + \epsilon \widehat{1(x_1)}
\widehat{K(2x_2)} + \widehat{K(2x_1)} \widehat{K(2x_2)}
\nonumber\\  & = & \delta(u_1) \delta(u_2) + \frac{\epsilon}{2}
I(u_1) \delta(u_2) + \frac{\epsilon}{2} \delta(u_1) I(u_2) +
\frac{\epsilon^2}{4}I(u_1) I(u_2) \nonumber\\ & & \mbox{where} \ \
I(u) = \ = \ \chi_{[-1,1]}(u)
\end{eqnarray}

It is straightforward to calculate the Fourier Transforms of the
above, as each function is even and of the form
$g_1(x_1)g_2(x_2)$. We also use the fact that $\widehat{g(2x)} =
\foh \widehat{g}(\frac{x}{2})$.

We now calculate $\widehat{W_{2,\epsilon,b}}(u)$. Note that $K$ is
even, so $K(x_i-x_j) = K(x_j-x_i)$.

\begin{eqnarray}
W_{2,\epsilon,b}(x) & = & \Big[K(x_1-x_2) + \epsilon K(x_1+x_2)
\Big] \Big[ K(x_2-x_1) + \epsilon K(x_1+x_2) \Big] \nonumber\\ & =
& K^2(x_1-x_2) + \epsilon^2 K^2(x_1+x_2) + 2 \epsilon K(x_1-x_2)
K(x_1+x_2) \nonumber\\ \widehat{W_{2,\epsilon,b}}(u) & = &
\widehat{T_-}(u_1,u_2) + \epsilon^2 \widehat{T_+}(u_1,u_2) + 2
\epsilon \widehat{T_3}(u_1,u_2)
\end{eqnarray}

Let $\eta = \pm 1$. Then

\begin{eqnarray}
\widehat{T_\eta}(u_1,u_2) = \int_{-\infty}^\infty
\int_{-\infty}^\infty K^2(x_1+\eta x_2) e^{2\pi i (u_1,u_2) \cdot
(x_1,x_2)} dx_1 dx_2.
\end{eqnarray}

Change variables: $t_1 = x_1 + \eta x_2, t_2 = x_2$. Then $x_1 =
t_1 -\eta t_2, x_2 = t_2$, and the Jacobian is $+1$.

\be \twovec{t_1}{t_2} \ = \ \mattwo{1}{\eta}{0}{1}
\twovec{x_1}{x_2}, \ \ \ \twovec{x_1}{x_2} \ = \
\mattwo{1}{-\eta}{0}{1}\twovec{t_1}{t_2}. \ee

Hence $dx_1dx_2 = dt_1dt_2$, and $(u_1,u_2) \cdot (x_1,x_2) =
u_1t_1 + (-\eta u_1 + u_2)t_2$. Hence

\begin{eqnarray}
\widehat{T_\eta}(u_1,u_2) & = & \int_{-\infty}^\infty
\int_{-\infty}^\infty K^2(t_1) e^{2 \pi i ( u_1t_1 + (-\eta u_1 +
u_2)t_2)}dt_1dt_2 \nonumber\\ & = & \int_{-\infty}^\infty K^2(t_1)
e^{2 \pi i u_1t_1} dt_1 \int_{-\infty}^\infty 1(t_2) e^{2 \pi i
(-\eta u_1 + u_2)t_2} dt_2 \nonumber\\ & = & \int_{-\infty}^\infty
K^2(t_1) e^{2 \pi i u_1t_1} dt_1 \cdot \delta(-\eta u_1 + u_2)
\end{eqnarray}

We have previously shown the Fourier Transform of $K^2(x_1)$ is
$(1 - |u_1|)I(u_1)$. We therefore find

\begin{eqnarray}
T_\eta(u_1,u_2) & = & \delta(-\eta u_1 + u_2) \cdot (1 - |u_1|
)\cdot I(u_1).
\end{eqnarray}

We calculate $\widehat{T_3}(u_1,u_2)$, the Fourier Transform of
$K(x_1-x_2)K(x_1+x_2)$. Change variables: $t_1 = x_1 - x_2$, $t_2
= x_1 + x_2$. Therefore $x_1 = \foh t_1 + \foh t_2$, $x_2 = -\foh
t_1 + \foh t_2$. The Jacobian is the absolute value of the
determinant of the transformation, which is $\foh$. In the
exponential we have $u_1x_1 + u_2x_2$, which becomes
$\foh(u_1-u_2)t_1 + \foh(u_1+u_2)t_2$.

\begin{eqnarray}
\widehat{T_3}(u_1,u_2) & = & \int \int K(x_1-x_2) K(x_1+x_2) e^{2
\pi i (u_1x_1 + u_2x_2)} dx_1 dx_2 \nonumber\\ & = & \int \int
K(t_1) K(t_2) e^{2 \pi i (\foh(u_1-u_2)t_1 + \foh(u_1+u_2)t_2)}
\frac{dt_1 dt_2}{2} \nonumber\\ & = & \foh \int K(t_1) e^{2 \pi i
\foh (u_1-u_2)t_1} dt_1 \int K(t_2) e^{2 \pi i \foh (u_1+u_2)t_2}
dt_2 \nonumber\\ & = & \foh \chiint(\frac{u_1-u_2}{2})
\chiint(\frac{u_1+u_2}{2}) \nonumber\\ & = & \foh I(u_1-u_2)
I(u_1+u_2),
\end{eqnarray}

where $I$ is the characteristic function of $[-1,1]$. If $|u_1| +
|u_2| > 1$, the above vanishes; $I$ is symmetric, and either $u_1
- u_2$ or $u_1 + u_2$ is $\pm(|u_1|+|u_2|)$. If $|u_1| + |u_2|
\leq 1$, the above is $1$. Hence

\begin{eqnarray}
\widehat{T_3}(u_1,u_2) = \foh I(|u_1|+|u_2|).
\end{eqnarray}

Collecting the pieces we obtain

\begin{eqnarray}
\widehat{W_{2,\epsilon}}(u) & = & \widehat{T_-}(u_1,u_2) +
\epsilon^2 \widehat{T_+}(u_1,u_2) + 2 \epsilon
\widehat{T_3}(u_1,u_2) \nonumber\\ & = & \delta(u_1+u_2)
\cdot(1-|u_1|)I(u_1) + \epsilon^2 \delta(-u_1+u_2)
\cdot(1-|u_1|)I(u_1) + \epsilon I(|u_1|+|u_2|).
\end{eqnarray}

We have proved

\begin{lem}[Expansion for $\widehat{W_{2,\epsilon}}(u)$]
Let $K(y) = \frac{\sin \pi y}{\pi y}$, $K_\epsilon(x,y) = K(x-y) +
\epsilon K(x+y)$,  $\epsilon = \pm 1$, and $W_{2,\epsilon}(x) =
\textbf{det}(K_\epsilon(x_i,x_j))$. We have
\begin{eqnarray}
\widehat{W_{2,\epsilon}}(u) & = & \widehat{W_{2,\epsilon,a}}(u) -
\widehat{W_{2,\epsilon,b}}(u) \nonumber\\ & = & \delta(u_1)
\delta(u_2) + \frac{\epsilon}{2} I(u_1) \delta(u_2) +
\frac{\epsilon}{2} \delta(u_1) I(u_2) + \frac{\epsilon^2}{4}I(u_1)
I(u_2) + \nonumber\\ & & \left[\delta(u_1+u_2) +
\gep^2\delta(-u_1+u_2)\right]\cdot(|u_1|-1)I(u_1) - \epsilon
I(|u_1|+|u_2|).
\end{eqnarray}
\end{lem}

$\\$ By duality, $\int \int f_1(x_1)f_2(x_2) W_{2,\epsilon}(x)
dx_1 dx_2 = \int \int \widehat{f_1}(u_1) \widehat{f_2}(u_2)
\widehat{W_{2,\epsilon}}(u) du_1 du_2$. Note (since $f_i$ is even)

\begin{eqnarray}
\int \int \hfo(u_1) \hft(u_2) \delta(\pm u_1+u_2)
\cdot(|u_1|-1)I(|u_1|)du_1du_2 = \int_{-1}^1 (|u|-1) \hfo(u)
\hft(u)du.
\nonumber\\
\end{eqnarray}

We simplify $\int \hfo(u)\hft(u)du$. \textbf{We assume the support
of each $\hfi$ is at most 1; this will still allow us to deal with
non-trivial regions, as we can have $|u_1|+|u_2| > 1$.} By duality
(for even functions), as $\hfi$ is supported in $(-1,1)$,

\begin{eqnarray}
\int_{-1}^1 \widehat{f_1}(u) \widehat{f_2}(u) du & = & \int
f_1(x)f_2(x)dx \nonumber\\ & = & \int (f_1f_2)(x) dx \nonumber\\ &
= & \widehat{f_1f_2}(0).
\end{eqnarray}

Therefore, for even functions of the form $f_1(x_1)f_2(x_2)$ whose
Fourier Transforms are supported in $|u_i|\leq 1$,

\begin{lem}\label{lemdensityepsilon}
\begin{eqnarray}
\int \int f_1(x_1)f_2(x_2) W_{2,\epsilon}(x) dx & = & \hfo(0)
\hft(0) + \frac{\epsilon}{2}f_1(0)\hft(0) +
\frac{\epsilon}{2}\hfo(0)f_2(0) + \frac{\epsilon^2}{4}f_1(0)f_2(0)
\nonumber\\ & & + \ (1+\epsilon^2) \int_{-1}^1 (|u|-1) \hfo(u)
\hft(u)du - \epsilon \int \int \hfo(u_1)\hft(u_2)I(|u_1|+|u_2|)du_1du_2 \nonumber\\
& = & \Big[\hfo(0) + \frac{\epsilon}{2} f_1(0) \Big] \Big[\hft(0)
+ \frac{\epsilon}{2} f_2(0) \Big] + \nonumber\\ & & (1+\epsilon^2)
\int_{-1}^1 |u| \hfo(u) \hft(u)du  -
(1+\epsilon^2)\widehat{f_1f_2}(0) \nonumber\\ & &- \epsilon \int
\int \hfo(u_1)\hft(u_2)I(|u_1|+|u_2|)du_1du_2.
\end{eqnarray}
\end{lem}

We could replace $\widehat{f_1f_2}(0)$ by leaving the $-1$ in the
$|u|$-integral:

\begin{eqnarray}
\int \int f_1(x_1)f_2(x_2) W_{2,\epsilon}(x) dx & = & \Big[\hfo(0)
+ \frac{\epsilon}{2} f_1(0) \Big] \Big[\hft(0) +
\frac{\epsilon}{2} f_2(0) \Big] + \nonumber\\ & & (1+\epsilon^2)
\int_{-1}^1 (|u|-1) \hfo(u) \hft(u)du  - \epsilon \int\int
\hfo(u_1)\hft(u_2)I(|u_1|+|u_2|)du_1du_2.\nonumber\\
\end{eqnarray}

For $SO(even)$, setting $\epsilon = 1$ above yields the 2-level
expansion, valid for any support.

\subsection{$2$-Level Densities}

We calculate the pieces needed to evaluate the densities (Equation
\ref{eqdensitykernels}).  We calculate \\ $\sum_{k=1}^2
\delta(x_k) \textbf{det} (K_{-1}(x_i,x_j))_{i,j\neq k}$; we've
already calculated $W_{2,\epsilon}(x) = \textbf{det}
(K_\epsilon(x_i,x_j))$.

\begin{eqnarray}
(W_{2,O^-})_{2}(x) & = & \sum_{k=1}^2 \delta(x_k)
\textbf{det}(K_{-1}(x_i,x_j))_{i,j\neq k} \nonumber\\ & = &
\delta(x_1) K_{-1}(x_2,x_2) + \delta(x_2) K_{-1}(x_1,x_1)
\nonumber\\ & = & \delta(x_1)\Big( 1 - K(2x_2) \Big) + \delta(x_2)
\Big( 1 - K(2x_1) \Big) \nonumber\\ & = & \delta(x_1) +
\delta(x_2) - \delta(x_1) K(2x_2) - \delta(x_2) K(2x_1)
\nonumber\\ & = & \delta(x_1)1(x_2) + 1(x_1)\delta(x_2) -
\delta(x_1) K(2x_2) - K(2x_1) \delta(x_2) \nonumber\\
\widehat{(W_{2,O^-})_{2}}(u) & = & 1(u_1)\delta(u_2) +
\delta(u_1)1(u_2) - \foh 1(u_1)I(u_2) - \foh I(u_1)1(u_2).
\nonumber\\
\end{eqnarray}

We determine the effect of $\widehat{(W_{2,O^-})_{2}}(u)$ on
$\widehat{f_1}(u_1)\widehat{f_2}(u_2)$ when $\mbox{supp}(f_i) <
1$.

\begin{eqnarray}
\int \int \widehat{f_1}(u_1)\widehat{f_2}(u_2)
\widehat{(W_{2,O^-})_{2}}(u) & = & \int \int
\hfo(u_1)\hft(u_2)1(u_1)\delta(u_2) + \int \int
\hfo(u_1)\hft(u_2)\delta(u_1)1(u_2) \nonumber\\ & & - \foh \int
\int \hfo(u_1)\hft(u_2)1(u_1)I(u_2) \nonumber\\ & & - \foh \int
\int \hfo(u_1)\hft(u_2) I(u_1)1(u_2) \nonumber\\ & = &
f_1(0)\widehat{f_2}(0) + \widehat{f_1}(0) f_2(0) - \foh
f_1(0)f_2(0) - \foh f_1(0)f_2(0) \nonumber\\ & = &
f_1(0)\widehat{f_2}(0) + \widehat{f_1}(0) f_2(0) - f_1(0)f_2(0).
\end{eqnarray}



\begin{thm}[$\mathcal{G} = SO(\mbox{even})$, $O$,
or $SO(\mbox{odd})$]\label{thmtwolevel}

Let $O^\gep$ represent $SO(\mbox{even})$ for $\gep = 1$ and
$SO(\mbox{odd})$ for $\gep = -1$. For even functions supported
with $\mbox{supp}(f_i) < 1$,
\begin{eqnarray}
\int \int \widehat{f_1}(u_1)\widehat{f_2}(u_2)
\widehat{W_{2,\mathcal{O}^\gep}}(u) du_1du_2 &= & \Big[\hfo(0) +
\foh f_1(0) \Big] \Big[\hft(0) + \foh f_2(0) \Big] \nonumber\\ & &
+ \ 2 \int_{-1}^1 (|u|-1) \hfo(u) \hft(u)du - \epsilon \int\int
\hfo(u_1)\hft(u_2)I(|u_1|+|u_2|)du_1du_2 \nonumber\\ & & - \
\frac{1-\gep}{2}f_1(0)f_2(0).
\end{eqnarray}
\end{thm}

\begin{rek} The term $\epsilon \int
\hfo(u_1)\hft(u_2)I(|u_1|+|u_2|)du_1du_2$ arises from $\int \int
f_1(x_1)f_2(x_2) K(x_1-x_2) K(x_1+x_2)dx_1dx_2$.
\end{rek}

\begin{rek} Remember $I(|u_1|+|u_2|) = I(u_1-u_2) I(u_1+u_2)$.
\end{rek}

For arbitrarily small support, the three $2$-level densities
differ. One increases by a factor of $\foh f_1(0) f_2(0)$ moving
from $\widehat{W_{2,O^+}}$ to $\widehat{W_{2,O}}$ to
$\widehat{W_{2,O^-}}$ when each $\hfi$ is supported in
$(-\foh,\foh)$.

\begin{rek} For comparison purposes, we record the 2-level moment
for SO(even). We assume $\hphi_1 = \hphi_2 = \hphi$ is supported
in $(-1,1)$. Thus, there are no combinatorial terms in the 2-level
density arising from odd elements (as everything is even). We have
to \emph{add back} twice the 1-level density with test function 
$\phi^2$ (this was because we were summing over $j_1 \neq \pm
j_2$); thus, we add back $2D_{1,SO(even)}(\phi^2)$.

If $\mbox{supp}(\hphi) = \sigma \in (\foh,1)$, then
$\mbox{supp}(\widehat{\phi^2}) > 1$. Thus, as $\sigma < 1$,

\bea D_{1,SO(even)}(\phi^2) & \ = \ & \widehat{\phi^2}(0) + \foh
\int \widehat{\phi^2}(u) I(u)du \nonumber\\ & = & \int \hphi(u)^2
I(u)du + \foh \int \widehat{\phi^2}(u) I(u)du. \eea

We subtract twice this, as well as the mean, $\left[ \hphi(0) +
\foh \phi(0)\right]^2$. We are left with the 2-level moment:

\bea 2 \int |u| \hphi(u)^2 I(u)du - \int\int \hphi(u)^2
I(|u_1|+|u_2|)du_1du_2 + \int \widehat{\phi^2}(u) I(u)du. \eea

\textbf{I believe the above is the centered second moment,
assuming $\hphi$ is supported in $(-1,1)$. For support in
$(-\foh,\foh)$, the last two terms add to zero. Further, the
middle term comes from the $2K(x_1-x_2) K(x_1+x_2)$ term; in your
notation, this was $2S(x_1-x_2) S(x_1+x_2)$.}

\end{rek}


To determine the density for $Sp$, we use Lemma
\ref{lemdensityepsilon} with $\epsilon = -1$. Rewriting the result
in a similar form as the orthogonal densities yields

\begin{thm}[$\mathcal{G} = Sp$] For even functions supported
with $\mbox{supp}(f_i) < 1$,
\begin{eqnarray}
\int \int \widehat{f_1}(u_1)\widehat{f_2}(u_2)
\widehat{W_{2,Sp}}(u) du_1du_2 &= & \Big[\hfo(0) - \frac{1}{2}
f_1(0) \Big] \Big[\hft(0) - \frac{1}{2} f_2(0) \Big] + \nonumber\\
& & 2 \int_{-1}^1 (|u|-1) \hfo(u) \hft(u)du  + \int
\hfo(u_1)\hft(u_2)I(|u_1|+|u_2|)du_1du_2.\nonumber\\
\end{eqnarray}
\end{thm}

To calculate $W_{2,U}(x)$, we need to take the determinant of

\begin{eqnarray}
\left(\begin{array}{cc}
             1 & \frac{\sin \pi(x_1-x_2)}{\pi(x_1-x_2)} \\
           \frac{\sin \pi(x_1-x_2)}{\pi(x_1-x_2)}  & 1
\end{array}\right)
\end{eqnarray}

Thus we need the Fourier Transform of $1 - \Big(\frac{\sin
\pi(x_1-x_2)}{\pi(x_1-x_2)}\Big)^2$. We find

\begin{thm}[$\mathcal{G} = U$]
\begin{eqnarray}
\widehat{W_{2,U}}(u) = \delta(u_1)\delta(u_2) -
\delta(u_1+u_2)\cdot (1-|u_1|)I(u_1).
\end{eqnarray}
Thus
\begin{eqnarray}
\int \int \widehat{f_1}(u_1)\widehat{f_2}(u_2) \widehat{W_{2,U}}
du_1du_2 &= & \hfo(0)\hft(0) + \int_{-1}^1 (|u|-1) \hfo(u)
\hft(u)du.
\end{eqnarray} \\
\end{thm}

Thus, for test functions with arbitrarily small support, the
$2$-level densities for the classical compact groups are mutually
distinguishable.



%!!!!!!!!!!!!!!!!!!!!!!!!!!!!!!!!!!!!!!!new section!!!!!!!!!!!!!!!!!!!!!!!!!!!!!!!!
%!!!!!!!!!!!!!!!!!!!!!!!!!!!!!!!!!!!!!!!new section!!!!!!!!!!!!!!!!!!!!!!!!!!!!!!!!
%!!!!!!!!!!!!!!!!!!!!!!!!!!!!!!!!!!!!!!!new section!!!!!!!!!!!!!!!!!!!!!!!!!!!!!!!!

\appendix

\section{Fourier Transform Simplifications}

Let $I(u)$ be the characteristic function of $[-1,1]$. If $K(x) =
\frac{\sin \pi x}{\pi x}$, note $\widehat{K(2x)}(u) = \foh I(u)$.
All functions below will be even Schwartz functions whose Fourier
Transforms have finite support. As all functions below are even,
we can use $\widehat{\widehat{f}}(u) = f(u)$, $\int \widehat{f}(u)
\widehat{g}(u)dt = \int f(x)g(x)dx$.

\begin{lem} \be \int_{u_1} I(|u_1|+|u_2|) e^{-2\pi i u_1 x_1}du_1
\ = \ \frac{\sin(2\pi(1-|u_2|)x_1)}{\pi x_1}. \ee \end{lem}

\bp This follows immediately from integrating. \ep

\begin{lem} \be \int_{u_2} \int_{u_1} \hphi(u_2)\hphi(u_1)
I(|u_1|+|u_2|)du_1du_2 \ = \ \int_u \widehat{\phi^2}(u)I(u)du. \ee
\end{lem}

\bp We have calculated the Fourier Transform of $I(|u_1|+|u_2|)$
with respect to $u_1$ above. Then

\bea \int_{u_2} \int_{u_1} \hphi(u_2)\hphi(u_1)
I(|u_1|+|u_2|)du_1du_2 & \ = \ & \int_{u_2}\hphi(u_2) \int_{x_1}
\phi(x_1) \frac{\sin(2\pi(1-|u_2|)x_1)}{\pi x_1} dx_1du_2. \ \ \
\eea

When we expand $\sin(2\pi(1-|u_2|)x_1)$, as $\hphi$ is even, we
need only keep the even term in $u_2$, $\sin(2\pi x_1) \cos(2\pi
|u_2|x_1)$. \textbf{This is wrong -- it is $|u_2|$ which appears,
not $u_2$; thus, we don't get a cancellation from oddness.} As
$\hphi$ is even, we may replace $\cos(2\pi|u_2|x_1)$ with
$e^{-2\pi i u_2 x_1}$. \textbf{What follows is just one of the
terms, from the sin cos.} Thus,

\bea \int_{u_2} \int_{u_1} \hphi(u_2)\hphi(u_1)
I(|u_1|+|u_2|)du_1du_2 & \ = \ & \int_{u_2} \hphi(u_2)\int_{x_1}
\phi(x_1) \frac{\sin(2\pi x_1)}{\pi x_1} e^{-2\pi i u_2x_1}
dx_1du_2 \nonumber\\ & = & 2\int_{x_1} \phi(x_1) \frac{\sin(2\pi
x_1)}{2\pi x_1} \int_{u_2} \hphi(u_2) e^{-2\pi i u_2 x_1} du_2
dx_1 \nonumber\\ & = & 2\int_{x_1} \phi(x_1) K(2x) \cdot \phi(x_1)
dx_1 \nonumber\\ & = & 2 \int_{x_1} \phi(x_1)^2 K(2x) dx_1 \ = \
\int_u \widehat{\phi^2}(u) I(u)du. \eea

\ep

\begin{lem}\label{lempsitwohat} $\widehat{\Psi_2}(x) = \phi(x)^2$ and
$\widehat{\Psi_3}(x) = \phi(x)^2$. \end{lem}

\bp The Fourier Transform converts convolution to multiplication:

\bea \Psi_2(u) & \ = \ & \int_w \hphi(w) \hphi(u-w)dw \nonumber\\
\widehat{\Psi_2}(x) & = & \int_u \int_w \hphi(w) \hphi(u-w)
e^{-2\pi i xu} dwdu \nonumber\\& = & \int_w \hphi(w) e^{-2\pi i
wx} dw \int_u \hphi(u) e^{-2\pi i xu}du \ = \ \phi(u)^2. \eea

Similarly $\widehat{\Psi_3}(x) = \widehat{\Psi_2}(x)\phi(x) =
\phi(x)^3$. \ep

\begin{lem}\label{lemwidehatphisquaredzero} $\widehat{\phi^2}(0) = \int \hphi(u)^2 I(u)du$.
\end{lem}

\bp As $\mbox{supp}(\phi) \subset (-1,1)$,

\bea \widehat{\phi^2}(0) \ = \ \int_x \phi(x)^2 dx  \ = \ \int_x
\phi(x) \phi(x) dx  \ = \ \int_u \hphi(u) \hphi(u)du  \ = \ \int_u
\hphi(u)^2 I(u)du. \eea \ep


\begin{acknowledgements}
Both authors thank AIM for its generous support; the second author
also wishes to thank Princeton and Ohio State.
\end{acknowledgements}

\begin{thebibliography}{PTW02} % '2nd argument contains the widest acronym'


\bibitem[CFKRS]{CFKRS}
\newblock B. Conrey, D. Farmer, P. Keating, M. Rubinstein and N.
Snaith, \emph{Integral Moments of $L$-Functions},
http://arxiv.org/pdf/math.NT/$0206018$

\bibitem[Da]{Da}
\newblock H. Davenport, \emph{Multiplicative Number Theory, $2$nd edition},
 Graduate Texts in Mathematics \textbf{74}, Springer-Verlag, New York,
 $1980$, revised by H. Montgomery.

\bibitem[De]{De}
\newblock P. Deligne, \emph{La conjecture de Weil. II} Inst. Hautes \'Etudes
Sci. Publ. Math. \textbf{52}, $1980$, $137-252$.

\bibitem[HW]{HW}
\newblock G. Hardy and E. Wright, \emph{An Introduction to the
Theory of Numbers}, fifth edition, Oxford Science Publications,
Clarendon Press, Oxford, $1995$.

\bibitem[Hej]{Hej}
\newblock D. Hejhal, \emph{On the triple correlation of zeros of
the zeta function}, Internat. Math. Res. Notices $1994$, no. $7$,
$294-302$.

\bibitem[HR]{HR}
\newblock C. Hughes and Z. Rudnick, \emph{Linear Statistics of
Low-Lying Zeros of $L$-functions}, to appear.

\bibitem[ILS]{ILS}
\newblock H. Iwaniec, W. Luo and P. Sarnak, \emph{Low lying zeros of
families of $L$-functions}, Inst. Hautes \'Etudes Sci. Publ. Math.
\textbf{91}, $2000$, $55-131$.

\bibitem[KS1]{KS1}
\newblock N. Katz and P. Sarnak, \emph{Random Matrices, Frobenius
Eigenvalues and Monodromy}, AMS Colloquium Publications
\textbf{45}, AMS, Providence, $1999$.

\bibitem[KS2]{KS2}
\newblock N. Katz and P. Sarnak, \emph{Zeros of zeta functions and symmetries},
Bull. AMS \textbf{36}, $1999$, $1-26$.

\bibitem[La1]{La1}
\newblock S. Lang, \emph{Algebraic Number Theory}, Graduate Texts in
Mathematics \textbf{110}, Springer-Verlag, New York, $1986$.

\bibitem[La2]{La2}
\newblock S. Lang, \emph{Real and Functional Analysis}, Graduate Texts in
Mathematics \textbf{142}, Springer-Verlag, New York, $1993$.

\bibitem[Mil1]{Mil1}
\newblock S. J. Miller, \emph{$1$- and $2$-Level Densities for Families of Elliptic
Curves: Evidence for the Underlying Group Symmetries}, P.H.D.
Thesis, Princeton University, $2002$,
http://www.math.princeton.edu/$\sim$sjmiller/thesis/thesis.pdf.

\bibitem[Mil2]{Mil2}
\newblock S. J. Miller, \emph{$1$- and $2$-Level Densities for Families of Elliptic
Curves: Evidence for the Underlying Group Symmetries}, Compositio
Mathematica, to appear.

\bibitem[Mon]{Mon}
\newblock H. Montgomery, \emph{The pair correlation of zeros of the zeta
function}, Analytic Number Theory, Proc. Sympos. Pure Math.
\textbf{24}, Amer. Math. Soc., Providence, $1973$, $181-193$.

\bibitem[Ru]{Ru}
\newblock M. Rubinstein, \emph{Evidence for a spectral
interpretation of the zeros of $L$-functions}, P.H.D. Thesis,
Princeton University, $1998$,
http://www.ma.utexas.edu/users/miker/thesis/thesis.html.

\bibitem[RS]{RS}
\newblock Z. Rudnick and P. Sarnak, \emph{Zeros of principal $L$-functions
 and random matrix theory}, Duke Journal of Math. \textbf{81},
 $1996$, $269-322$.


\end{thebibliography}

\end{document}
