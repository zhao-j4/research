Katz and Sarnak conjectured a
correspondence between the $n$-level density statistics of zeros from
families of $L$-functions
with eigenvalues from random matrix ensembles, and in many cases the sums of smooth test functions, whose Fourier transforms are
finitely supported over scaled zeros in a family, converge to an integral of
the test function against a density $W_{n, G}$ depending on the
symmetry $G$ of the family (unitary, symplectic or orthogonal). This integral bounds the average order of vanishing at the central point of the corresponding family of $L$-functions. 

We can obtain better estimates on this vanishing in two ways. The first is
to do more number theory, and prove results for larger $n$ and greater
support; the second is to do functional analysis and obtain better test
functions to minimize the resulting integrals. We pursue the latter here
when $n=2$, minimizing  
	\[ \frac{1}{\Phi(0, 0)} \int_{{\mathbb R}^2} W_{2,G} (x, y) \Phi(x, y) dx dy \] 
over test functions $\Phi : {\mathbb R}^2
\to [0, \infty)$ with compactly supported Fourier transform. We study a
restricted version of this optimization problem, imposing that our test
functions take the form $\phi(x) \psi(y)$ for some fixed admissible $\psi(y)$ and
$\supp{\widehat \phi} \subseteq [-1, 1]$. Extending results from the 1-level case, namely the functional analytic arguments of Iwaniec, Luo and Sarnak and the differential equations method introduced by Freeman and Miller, we explicitly solve for the optimal $\phi$ for appropriately chosen fixed test function $\psi$. We conclude by discussing further improvements on estimates  by method of iteration.