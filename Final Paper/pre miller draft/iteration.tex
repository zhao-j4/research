One can improve the bounds found in Table \ref{tab:bounds} by choosing $\phi_{G, \psi}$ as our new fixed test function and optimizing accordingly. As $\widehat{\phi_{G, \psi}}$ takes the form of a piecewise trigonometric polynomial for each of the classical compact groups, the methods used in Section \ref{sec:quadker} are not applicable. We instead appeal to the standard approach to solving Fredholm integral equations by method of iteration. Suppose an even continuous kernel $m: [-1, 1] \to \R$ satisfies
	\begin{equation}
		\int_{-1/2}^{1/2} \int_{-1/2}^{1/2} |m(x - y)|^2 dx dy < 1. 
	\end{equation}
Define a self-adjoint compact operator $K: L^2 [-1/2, 1/2] \to L^2[-1/2, 1/2]$ by
	\begin{equation}
		(Kf)(x) := \int_{-1/2}^{1/2} m(x - y) f(y) dy.
	\end{equation}
It follows from the Cauchy-Schwarz inequality that the operator norm satisfies $||K||_{L^2 \to L^2} < 1$, that is, $K$ is a contraction mapping, since
	\begin{equation}
		||Kg||_2^2 = \int_{-1/2}^{1/2} \left( \int_{-1/2}^{1/2} m(x - y) g(y) dy \right)^2 dx \leq ||g||_2^2 \int_{-1/2}^{1/2} \int_{-1/2}^{1/2} |m(x - y)|^2 dx dy < 1. 
	\end{equation}	
Thus by the Weierstrass $M$-test, the series
	\begin{equation}
		g(x) := \sum_{n = 0}^\infty (-1)^n K^n (1)(x)
	\end{equation}
converges absolutely and uniformly on the interval $[-1/2, 1/2]$. Moreover, it is the unique continuous solution to the Fredholm integral equation $(I + K) g = 1$. Since the series converges absolutely, we can integrate term by term to obtain the corresponding minimum value,
	\begin{equation}
		\frac{c_{G, \phi}}{\langle 1, g \rangle} = c_{G, \phi}\left( \sum_{n = 0}^\infty (-1)^n \int_{-1/2}^{1/2} K^n (1) (x) dx \right)^{-1}. \label{eq:seriesmin}
	\end{equation}
Unfortunately, this method of deriving new bounds is computationally intensive as we need to compute $n$-dimensional integrals of unwieldy expressions. Additionally, depending on the rate of convergence we may need to compute a large number of terms to obtain meaningful degrees of accuracy. For the purposes of this paper we focus on the unitary group, where these challenges can be avoided. 

For brevity, denote $\phi := \phi_{\Unit, \psi}$. In this case we know the series converges, as
\begin{align}
		\int_{-1/2}^{1/2} \int_{-1/2}^{1/2} |m_{\Unit, \phi}(x - y)|^2 dx dy = \frac{2 \sin^2 (1) (128 - 110 \cos (2) - 37 \sin (2))}{3 (-8 + 6 \cos (2) - \sin (2))^2} < 1.
\end{align}
Moreover, $\widehat{\phi}$ is non-negative, so it follows that $(-1)^n K^n_{\Unit, \phi}(1)$ is non-negative. We can therefore truncate the series in (\ref{eq:seriesmin}) to obtain a legitimate upper bound, as the terms are non-negative. Summing five terms gives the bound
	\begin{equation}
		\inf_\Phi \frac{1}{\Phi(0, 0)} \int_{\R^2} \Phi(x, y) W_{2, \Unit} (x, y) dx dy \leq c_{G, \psi}\left( \sum_{n = 0}^5 (-1)^n \int_{-1/2}^{1/2} K^n (1) (x) dx \right)^{-1} \approx 0.4888,
	\end{equation}
a small improvement on our previous bound in Table \ref{tab:bounds}.